%\documentclass[12pt]{article}
%\usepackage{CJK}
%\usepackage{pinyin}

\begin{CJK}{Bg5}{bsmi}

%---------------------------------------------
%	Chapter Introduction
%---------------------------------------------


\chapter{Introduction}
% 現在web service越來越多,而authentication system是其中重要的一環。(為什麼? 可以舉一些被攻擊的例子以及後果)
In the age of information, more and more services are provided through the Internet.
For these web services, it is important that how to verify users' identity on the web.
Therefore, a secure \emph{Authentication System} is essential for each service.
Nowadays, the password-based scheme is the most common authenication system.

\section{Motivation}
% 目前有很多方法被提出來,不過有些方法已經有攻擊的方法。 舉一些攻擊的手法加ref citation
Lots of researchers have demonstrate various authenication methods in recent decades, but some of them are not secure enough.
Malicious attackers can get people's password in many different approaches.
Moreover, attackers can even steal the private information of users.
\emph{For example, 方法1 can get the password through blah blah.
Besides, 方法二 can blah blah. In brief, there are several security weaknesses in the current systems.
In brief, several systems have severe security issue.
}

% 除了安全的issue以外,有些方法的usability很差, 一樣舉例
\emph{
In addition to the security issue, some system has lower usability. It means that blah blah blah. For instance, XXX system need to use blah blah, and blah blah. In conclusion, these system is hard to use.
}

% 因為以上的原因,這篇paper要提出一個兼顧security和usability的system.
% 為什麼別人的無法兼顧而你的可以呢? 把你take advantage的東西大致講一下。
\emph{
According to these reason mensioned above, the aim of the thesis is to develop a authentication system with high security and usability. 
% security
I use blah blah to prevent the security issue.
% usability
Then I take advantage of blah blah to build a user-friendly system.
% Conclusion
Compared to existed system, our authentication system is much secure and usable.
}

The rest of the thesis is organized as follows.
In chapter 3, I describe this system more detailed, and give a high-level code example to explain the implementation.
In chapter 4, I build some criteria to estimate the performance of my authentication system in security, deployment ability, and usability.
In the last chapter, I compare the existing authentication schemes to my new design, and the proposed system is much useful by comparison.


\end{CJK}
%%% Local Variables: 
%%% mode: latex
%%% TeX-master: "paper"
%%% End: 

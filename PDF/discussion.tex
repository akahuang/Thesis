\begin{CJK}{Bg5}{bsmi}

%---------------------------------------------
%	Chapter Discussion
%---------------------------------------------

\chapter{Discussion}

\section{Security Analysis}

\begin{comment}
This is the most important part of an authentication system.
We have to define our threat model before we start to analyze.
There are 4 components in the scheme I proposed.
\end{comment}

\subsection{Threat Model}

\subsection{Analysis}

\section{Usability Analysis}

\subsection{Criterias}

\subsubsection{Memorywise-Effortless}

a

\subsubsection{Scalable-for-user}

b

\subsubsection{Nothing-to-carry}

c

\subsubsection{Physical-Effortless}

d

\subsubsection{Easy-to-learn}

e

\subsubsection{Efficiency}

f

\subsubsection{Infrequent-Error}

g

\subsubsection{Easy-recovery-from-loss}

\subsection{Analysis}

\begin{comment}
The usability can not be neglected when researchers trying to design a system.
Usability is a subjective perception, it may be different from person to person.
The following paragragh states the criterias I used to estimate the usability of a system.
\end{comment}

\section{Deployment Ability Analysis}

\subsection{Criterias}

\subsubsection{Accessible}

a

\subsubsection{User-cost}

b

\subsubsection{Server-compatable}

c

\subsubsection{Browser-compatable}

d

\subsubsection{Mature}

e

\subsection{Analysis}

\begin{comment}
The deployment ability is also an important thing which is need to be considered, especially in designing an authentication system.
A system with high usability means it is friendly to users; a system with high deployment ability means it is friendly to the system provider or, more precise, the developers.
The following paragraph states the criterias I used to estimate a system's deployment ability. 
\end{comment}


\end{CJK}
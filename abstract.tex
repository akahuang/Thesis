%\documentclass[12pt]{report}

%\usepackage[a4paper,left=3cm,top=3cm,right=3cm,bottom=2cm,nohead]{geometry}

\doublespacing

%\begin{document}
\onehalfspacing

\begin{titlepage}
\begin{CJK}{UTF8}{bkai}
\begin{center}
\Large{{摘要}}\\
\end{center}

    廣受使用的NTRU密碼系統和許多可証明安全性的格基密碼系統的安全性都直接和最短格向量問題有高度相關。於此研究中,我們改進數點解最短格向量問題之演算法,至今,經由這些改進,在解SVP的公開挑戰賽中我們分別拿下第一、二、三名的成績。

    我們改進目前效率最高的極速列舉演算法,並分別經由MapReduce實作在雲端上和經由CUDA實作在顯示晶片上。藉由這兩個實作,可以價錢來評估一格基密碼系統的難度,意即,要花多少錢向雲端計算提供商租運算量去破解一格基密碼系統。

    我們的實作使用8張nVidia顯卡即可在兩天內破解維度114,116的最短格向量問題;另外,我們也花費2300美金解出維度120的問題。




\vspace*{5em}

{關鍵字:} \emph{格基密碼學、最短格向量問題、顯示晶片、雲端運算、列舉演算法、極速列舉演算法。}


\end{CJK}
\end{titlepage}

%\end{document}
